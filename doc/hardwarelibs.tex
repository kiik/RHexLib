%
% This file is part of RHexLib, 
%
% Copyright (c) 2001 The University of Michigan, its Regents,
% Fellows, Employees and Agents. All rights reserved, and distributed as
% free software under the following license.
% 
%  Redistribution and use in source and binary forms, with or without
% modification, are permitted provided that the following conditions are
% met:
% 
% 1) Redistributions of source code must retain the above copyright
% notice, this list of conditions, the following disclaimer and the
% file called "CREDITS" which accompanies this distribution.
% 
% 2) Redistributions in binary form must reproduce the above copyright
% notice, this list of conditions, the following disclaimer and the file
% called "CREDITS" which accompanies this distribution in the
% documentation and/or other materials provided with the distribution.
% 
% 3) Neither the name of the University of Michigan, Ann Arbor or the
% names of its contributors may be used to endorse or promote products
% derived from this software without specific prior written permission.
% 
% THIS SOFTWARE IS PROVIDED BY THE COPYRIGHT HOLDERS AND CONTRIBUTORS
% "AS IS" AND ANY EXPRESS OR IMPLIED WARRANTIES, INCLUDING, BUT NOT
% LIMITED TO, THE IMPLIED WARRANTIES OF MERCHANTABILITY AND FITNESS FOR
% A PARTICULAR PURPOSE ARE DISCLAIMED. IN NO EVENT SHALL THE REGENTS OR
% CONTRIBUTORS BE LIABLE FOR ANY DIRECT, INDIRECT, INCIDENTAL, SPECIAL,
% EXEMPLARY, OR CONSEQUENTIAL DAMAGES (INCLUDING, BUT NOT LIMITED TO,
% PROCUREMENT OF SUBSTITUTE GOODS OR SERVICES; LOSS OF USE, DATA, OR
% PROFITS; OR BUSINESS INTERRUPTION) HOWEVER CAUSED AND ON ANY THEORY OF
% LIABILITY, WHETHER IN CONTRACT, STRICT LIABILITY, OR TORT (INCLUDING
% NEGLIGENCE OR OTHERWISE) ARISING IN ANY WAY OUT OF THE USE OF THIS
% SOFTWARE, EVEN IF ADVISED OF THE POSSIBILITY OF SUCH DAMAGE.

%%%%%%%%%%%%%%%%%%%%%%%%%%%%%%%%%%%%%%%%%%%%%%%%%%%%%%%%%%%%%%%%%%%%%%
% $Id: hardwarelibs.tex,v 1.2 2001/07/19 23:37:43 ulucs Exp $
%
% Created       : Uluc Saranli, 01/06/2001
% Last Modified : Uluc Saranli, 06/27/2001
%
%%%%%%%%%%%%%%%%%%%%%%%%%%%%%%%%%%%%%%%%%%%%%%%%%%%%%%%%%%%%%%%%%%%%%%

\chapter{Hardware Libraries}
\label{sec:hardware_libraries}

\section{RHex Michigan Hardware}

\subsection{The {\tt dm6814} Class}

\subsection{The {\tt mpc550} Class}

\subsection{{\tt MEncoder} Component}

\subsection{{\tt MAnalog} Component}

\subsection{{\tt MDigital} Component}

\subsection{{\tt MTimer} Component}

\subsection{{\tt MAccel} Component}

\subsection{{\tt MPower} Component}

\subsection{{\tt MSwitch} Component}

\subsection{{\tt MDial} Component}

\subsection{{\tt MMotor} Component}

{\tt MMotor} class is the DCMotorHW component of the Michigan hardware
implementation. It has the following features:

\begin{itemize}
\item{Estimation of motor current, motor terminal voltage and motor back EMF
using a crude model of the APEX drive voltage drop.}

\item{APEX drive voltage drop compensation in setting the terminal voltage
command to the motor.}

\end{itemize}

The estimated motor values are computed only once per cycle. This is
accomplished by checking the last read values of the voltage command setting
and the encoder speed reading and performing recomputations only when these
values change.\\

\noindent{\bf Motor state estimation algorithms}\\

The motor back EMF is computed using the latest motor speed measurement.

\begin{equation*}
V_E = \frac{\omega}{K_\omega k_g}
\end{equation*}

\noindent where $\omega$ is the leg rotation speed ($rad/s$), $K_\omega$ is
the motor speed constant ($(rad/s)/V$) and $k_g$ is the gear ratio (output
rev/shaft rev).

The outputs of the APEX motor drives are pulse-width modulated (PWM)
signals, whose duty factor is an affine function of the command signal to
the motor drives. Note that this command signal is different than the
voltage across motor terminals.

\begin{equation}
\phi = s_d (V_c - V_c^0)
\label{eq:affine_apex_command}
\end{equation}

\noindent where $\phi$ is the duty factor in the range $[-1, 1]$,
$s_d$ is the drive command scaling factor, also incorporating the polarity
of the motor drive ($1/Volts$), $V_c$ is the latest command output to the APEX
drive and $V_c^0$ is the neutral command, resulting in $\phi = 0$
(empirically determined by the calibration procedures).

The final step in the estimation is to find the operating point for the
motor terminal voltage and the armature current. Two equations govern the
operation of the motor.

\begin{eqnarray*}
V_o & = & r_a I + V_E \\
V_o & = & \phi ( V_s - r_d I )
\end{eqnarray*}

\noindent where $V_o$ and $I$ are the armature voltage and current, $V_s$ is
the supply voltage, $r_a$ is the armature resistance and $r_d$. Under this
simple model, the operating point for the current can be easily found.

\begin{equation*}
I = \frac{\phi V_s - V_E}{r_a + \phi r_d}
\end{equation*}

However, this model has serious problems. Note that for $\phi < 0$, the
drive drop model starts acting like an active element. As a consequence,
when $\phi = -r_a/r_d$, the current becomes infinite. We solve this problem
by making sure that the drive drop (which, in fact, is a consequence of the
internal resistance of the switching elements) remains a passive element by
using the following drive drop model.

\begin{equation*}
V_o  =  \phi V_s - |\phi| r_d I
\label{eq:drive_model}
\end{equation*}

Hence, the final motor state estimation equations take the following form.

\begin{eqnarray*}
V_E & = & \frac{\omega}{K_\omega k_g} \\
\phi & = & s_d (V_c - V_c^0) \\
I & = & \frac{\phi V_s - V_E}{r_a + |\phi| r_d} \\
V_o & = & r_a I + V_E
\end{eqnarray*}

These equations are implemented in the {\tt MMotor::computeStates()}
method. These computations can be disabled by setting the configuration
symbol {\tt motor\_estimation\_enable} to 0. 
\symbolindex{\tt motor\_estimation\_enable}.\\

\noindent{\bf APEX voltage drop compensation in setting the voltage command}\\

The DCMotorHW interface requires that the {\tt setCommand()} method's
argument specifies the desired {\tt terminal} voltage. The simplest model of
the motor drives would assume that there is no supply voltage drop as a
consequence, resulting in the following equation to compute the drive
command.

\begin{equation*}
V_c = \frac{1}{s_d} \frac{V_o}{V_s} + V_c^0
\end{equation*}

\noindent where $V_o$ is the desired terminal voltage and $s_d$ is the drive
scaling coefficient as defined above. However, the voltage drop at the motor
drives is usually very large and cannot be ignored. If it is ignored, the
actual motor terminal voltage would end up being lower in magnitude than the
desired value, resulting in degradation of the approximate torque
control. As a consequence, the Michigan hardware implementation features 
APEX drive drop compensation when setting the terminal voltage through the
{\tt setCommand()} interface.

Let's assume that we are given a desired terminal voltage $V_o$. We will
need to solve the following equation for $\phi$ in order to determine the
command that we need to apply to the APEX drives (which is an affine
function of $\phi$ as in (\ref{eq:affine_apex_command})) following equation.

\begin{equation*}
V_o = \phi V_s - |\phi| r_d (V_o - V_E)/r_a
\end{equation*}

At this point, we also make the assumption that $V_s - r_d(V_o - V_E)/r_a >
0$. This is a reasonable assumption because it is physically impossible for
the voltage drop on the motor drives to result in a reversal of the supply
voltage behavior. The voltage drop can at most drop the voltage to zero,
beyond which our model does not claim to have any validity anyway.

Assuming $\phi > 0$, we obtain the following solution

\begin{equation*}
\phi = \frac{V_o}{V_s - r_d(V_o - V_E)/r_a}
\end{equation*}

\noindent which shows that $\phi$ has the same sign as $V_o$. Similarly,
assuming $\phi < 0$ yields
 
\begin{equation*}
\phi = \frac{V_o}{V_s + r_d(V_o - V_E)/r_a}
\end{equation*}

\noindent where $\phi$, once again has the same sign as $V_o$. As a
consequence, the sign of $\phi$ is always the same as the sign of $V_o$. we
cannow write down a closed form solution for the duty factor solely as a
function of the desired command voltage. Note that we also need to clip the
duty factor in the range $[-1, 1]$ to model the impossibility of supplying a
terminal voltage larger than the effective supply voltage after the drive
drop.

\begin{eqnarray*}
\phi_u & = &\left\{\begin{array}{ll}
V_o / ( V_s - r_d(V_o - V_E)/r_a)
& \text{\hspace{0.1in}if\hspace{0.1in}}V_o > 0 \\
V_o / ( V_s + r_d(V_o - V_E)/r_a)
& \text{\hspace{0.1in}if\hspace{0.1in}}V_o < 0
\end{array}\right. \\
\phi & = & \mathrm{max}(\ \mathrm{min}(\ 1, \phi_u\ ), -1\ )
\end{eqnarray*}

Finally, the command voltage for the APEX drive is computed using the affine
map determined after the calibration procedure.

\begin{equation*}
V_c = \frac{\phi}{s_d}  + V_c^0
\end{equation*}

This model is implemented in the {\tt MMotor::setCommand()} method of the
Michigan hardware library. Moreover, this compensation can be disabled by
setting the {\tt drive\_compensation\_enable} symbol to 0.
\symbolindex{\tt drive\_compensation\_enable}
\section{SimSect Hardware}

\section{Virtual Hardware}
