%
% This file is part of RHexLib, 
%
% Copyright (c) 2001 The University of Michigan, its Regents,
% Fellows, Employees and Agents. All rights reserved, and distributed as
% free software under the following license.
% 
%  Redistribution and use in source and binary forms, with or without
% modification, are permitted provided that the following conditions are
% met:
% 
% 1) Redistributions of source code must retain the above copyright
% notice, this list of conditions, the following disclaimer and the
% file called "CREDITS" which accompanies this distribution.
% 
% 2) Redistributions in binary form must reproduce the above copyright
% notice, this list of conditions, the following disclaimer and the file
% called "CREDITS" which accompanies this distribution in the
% documentation and/or other materials provided with the distribution.
% 
% 3) Neither the name of the University of Michigan, Ann Arbor or the
% names of its contributors may be used to endorse or promote products
% derived from this software without specific prior written permission.
% 
% THIS SOFTWARE IS PROVIDED BY THE COPYRIGHT HOLDERS AND CONTRIBUTORS
% "AS IS" AND ANY EXPRESS OR IMPLIED WARRANTIES, INCLUDING, BUT NOT
% LIMITED TO, THE IMPLIED WARRANTIES OF MERCHANTABILITY AND FITNESS FOR
% A PARTICULAR PURPOSE ARE DISCLAIMED. IN NO EVENT SHALL THE REGENTS OR
% CONTRIBUTORS BE LIABLE FOR ANY DIRECT, INDIRECT, INCIDENTAL, SPECIAL,
% EXEMPLARY, OR CONSEQUENTIAL DAMAGES (INCLUDING, BUT NOT LIMITED TO,
% PROCUREMENT OF SUBSTITUTE GOODS OR SERVICES; LOSS OF USE, DATA, OR
% PROFITS; OR BUSINESS INTERRUPTION) HOWEVER CAUSED AND ON ANY THEORY OF
% LIABILITY, WHETHER IN CONTRACT, STRICT LIABILITY, OR TORT (INCLUDING
% NEGLIGENCE OR OTHERWISE) ARISING IN ANY WAY OUT OF THE USE OF THIS
% SOFTWARE, EVEN IF ADVISED OF THE POSSIBILITY OF SUCH DAMAGE.

%%%%%%%%%%%%%%%%%%%%%%%%%%%%%%%%%%%%%%%%%%%%%%%%%%%%%%%%%%%%%%%%%%%%%%
% $Id: lacvshelp.tex,v 1.2 2001/07/12 17:14:10 ulucs Exp $
%
% CVS tutorial documentation.
%
% Created       : Laura McWilliams, 05/12/2001
% Last Modified : Laura McWilliams, 05/27/2001
%
%%%%%%%%%%%%%%%%%%%%%%%%%%%%%%%%%%%%%%%%%%%%%%%%%%%%%%%%%%%%%%%%%%%%%%

\documentclass[12pt, letterpaper]{article}

\usepackage{hhline}
\usepackage{alltt}

\setlength{\paperwidth}{8.5in}
\setlength{\paperheight}{11in}


\begin{document}


\title{\bf {\Large Laura's Abbreviated Guide to CVS}\\ 
           {\large (for new Kodlab members)}}
\author{\large \it Laura McWilliams (lmcwil@umich.edu)}
\maketitle 

\section*{Introduction}
These are the steps you must follow to start using CVS in
Kodlab. Before executing any CVS commands, you must ssh to poyraz,
because that is where the CVS repository is located. So before you do
any of the following commands, first type:\\
{\bf ssh poyraz}\\
then enter your password when prompted. To exit leave the secure shell 
(ssh), simply type:\\
{\bf logout}\\
I don't think you are stupid, I'm just telling you these things in
case you don't know :)



\section{Setting stuff up to use CVS}
  \subsection{}
You'll want to set up a text editor because whenever you 
have to enter a CVS log, it needs to know what editor to open up for
you. I'd recommend xEmacs. Put this line in your .cshrc file:\\
      {\bf setenv EDITOR xemacs }\\
Or, alternatively, you can simply type the above line from the shell
which you are going to be in when you use cvs, every time. 
  \subsection{}
Another line you need to either enter at the shell or
put in your .cshrc file is this:\\
      {\bf setenv CVSROOT uniqname@yildiz:/home/uniqname/dirname }\\
      where dirname is the name of the directory containing the
      directories which you are keeping track of with CVS.
  \subsection{} 
One more environmental variable to set:\\
      {\bf setenv CVS\_RSH ssh }



\section{Checking out somebody else's directory (such as RHexLib)}
  \subsection{} 
Go to the directory where you want this new directory to
reside. Do {\it not} create the directory. Type the following line:\\
      {\bf cvs -d /home/rhex/cvs checkout dirname }\\
where dirname is the name of the directory in the
repository. This will also become the name of your directory, which
will be added by CVS. NOTE: if you cd to /home/rhex/cvs you can see
which different directories are available to be checked out.



\section{Creating your own repository with CVS}
  \subsection{} 
Go {\it into } the actual directory you want to add and type the
following line:\\
      {\bf cvs -d /home/rhex/cvs import dirname uniqname start }\\
where dirname is the name you desire for your directory, and
uniqname is your uniqname. NOTE: You should first clean out the
directory, removing any files you {\it don't } want added to the
repository, such as object files.



\section{Keeping your repository updated}
If you are working with a directory that other people are working on,
it's wise to update it before you start working, since this reduces
the risk of making changes that conflict with other people's
changes. If you want your changes to affect the repository, you must
commit them. Until you have actually committed your directory, no one
can see the changes you've made but you.
  \subsection{} 
To update your directory, go {\it into} that directory 
and type:\\
      {\bf cvs update }\\
Whenever you update, CVS gives you a bunch of messages that show you
which files have changed and it will warn you if there are possible
conflicts (more on that later). So it's not a bad idea to actually
look at the messages to see which files have been updated (have a U
next to them) and {\it especially} which have been merged (have a M
next to them).
  \subsection{} 
To commit your directory, go {\it into} that directory 
and type:\\
      {\bf cvs commit }\\
      You will be prompted to enter a log for the changes that you
have made. Simply save and exit the file it opened up for you after
you have entered your message. If you have made a bunch of changes,
but you only want to enter one log message, it will automatically open 
the same file up for you, so you can simply exit it again. It will ask 
you if you meant to do this; one of the options is to use the
same message for all the log entries, so you can pick this option.

\section{Adding and removing files}
  \subsection{} 
To add a single file to a repository, go into that
directory (where the file is already located in your own individual
copy) and type:\\
      {\bf cvs add filename }\\
  \subsection{} 
To add a directory and all of its contents, do the same thing:\\
      {\bf cvs add dirname }\\
  \subsection{} 
To remove a single file, first delete the file in your
directory. Then type:\\
      {\bf cvs remove filename }\\
  \subsection{} 
To remove an entire directory, first delete all of the
individual files in that directory, without deleting the directory itself. Then type:\\
      {\bf cvs remove -R dirname }\\
      (Make sure you use a capital R.)Afterward, you can remove the
directory itself from your own directory.

\section{Pitfalls, other features, etc}
When a lot of people are modifying the same library, there are a lot
of chances for things to get screwed up. Fortunately, CVS can save you 
a lot of headaches.
  \subsection{} 
To see the difference between your version of a file and 
the version in the CVS repository, type:\\
      {\bf cvs diff filename }\\
  \subsection{Merged files}
Merged files (files that will have an M next to them when you do {\bf
update}) are important to watch for. Whenever CVS notices that you
have made changes as well as somebody else since the last time you
committed, it merges the two files together line for line. Oftentimes, 
you'll be working on a different section of the code than the other
person was, and there will be no problem. However, it's a good idea to 
check over the code to make sure nothing weird happened.
When CVS cannot manage to successfully merge the files on its own, it
outputs a message to warn you. It also goes into the file in question
and marks where the problem is with arrow-brackets, like this:\\
{\bf $>>>>>>>>>$ }\\
or something. So look for those marks, and you'll see that CVS has
taken the lines in question and put both versions. You'll have to
manually merge them yourself if you want the code to actually work.

\section*{Conclusion}
CVS is a really neat tool and a necessity for us to keep RHexLib in
order. Plus, if you totally screw up, it's a nice safety net, since
CVS hangs onto all these old versions of the project. Well, I hope
you've found this little sheet helpful.

\end{document}
