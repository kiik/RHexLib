%
% This file is part of RHexLib, 
%
% Copyright (c) 2001 The University of Michigan, its Regents,
% Fellows, Employees and Agents. All rights reserved, and distributed as
% free software under the following license.
% 
%  Redistribution and use in source and binary forms, with or without
% modification, are permitted provided that the following conditions are
% met:
% 
% 1) Redistributions of source code must retain the above copyright
% notice, this list of conditions, the following disclaimer and the
% file called "CREDITS" which accompanies this distribution.
% 
% 2) Redistributions in binary form must reproduce the above copyright
% notice, this list of conditions, the following disclaimer and the file
% called "CREDITS" which accompanies this distribution in the
% documentation and/or other materials provided with the distribution.
% 
% 3) Neither the name of the University of Michigan, Ann Arbor or the
% names of its contributors may be used to endorse or promote products
% derived from this software without specific prior written permission.
% 
% THIS SOFTWARE IS PROVIDED BY THE COPYRIGHT HOLDERS AND CONTRIBUTORS
% "AS IS" AND ANY EXPRESS OR IMPLIED WARRANTIES, INCLUDING, BUT NOT
% LIMITED TO, THE IMPLIED WARRANTIES OF MERCHANTABILITY AND FITNESS FOR
% A PARTICULAR PURPOSE ARE DISCLAIMED. IN NO EVENT SHALL THE REGENTS OR
% CONTRIBUTORS BE LIABLE FOR ANY DIRECT, INDIRECT, INCIDENTAL, SPECIAL,
% EXEMPLARY, OR CONSEQUENTIAL DAMAGES (INCLUDING, BUT NOT LIMITED TO,
% PROCUREMENT OF SUBSTITUTE GOODS OR SERVICES; LOSS OF USE, DATA, OR
% PROFITS; OR BUSINESS INTERRUPTION) HOWEVER CAUSED AND ON ANY THEORY OF
% LIABILITY, WHETHER IN CONTRACT, STRICT LIABILITY, OR TORT (INCLUDING
% NEGLIGENCE OR OTHERWISE) ARISING IN ANY WAY OUT OF THE USE OF THIS
% SOFTWARE, EVEN IF ADVISED OF THE POSSIBILITY OF SUCH DAMAGE.

%%%%%%%%%%%%%%%%%%%%%%%%%%%%%%%%%%%%%%%%%%%%%%%%%%%%%%%%%%%%%%%%%%%%%%
% $Id: faq.tex,v 1.5 2001/07/19 16:35:56 ulucs Exp $
%
% Created       : Uluc Saranli, 01/06/2001
% Last Modified : Uluc Saranli, 06/27/2001
%
%%%%%%%%%%%%%%%%%%%%%%%%%%%%%%%%%%%%%%%%%%%%%%%%%%%%%%%%%%%%%%%%%%%%%%

\chapter{Frequently Asked Questions}

\section{Installation}

\begin{enumerate}
\item
\end{enumerate}

\section{Makefiles and Compiling}

\begin{enumerate}
\item
\end{enumerate}

\section{Module Manager and Modules}

\begin{enumerate}
\item{In the function call {\tt MMGrabModule(moduleptr,this)}, what does the
    {\tt this} keyword do?}\par
The C++ keyword {\tt this} is a pointer to the class object for which the
currently executing method was called. In this case, it points to the
current module.

\item{What happens when I try to read more than one configuration file with
    {\tt MMReadConfigFile()}?}\par
The contents of the new configuration file are appended to the existing
symbol table. If a symbol is redefined, the new value replaces the old one.

\item{What happens if two modules have the same order value?}\par
The order in which their update functions are called in the same module
manager cycle is unspecified. It may depend on the time of the day, the
weather, barometric pressure etc. etc.
\end{enumerate}

\section{State Machines}

\begin{enumerate}
\item{Why do state machines have no {\tt update()} method?}\par

The \StateMachine\ class is derived from the \Module\ class and it defines
the update function itself. It performs various importnat tasks such as
checking the currently active events, calling the during() function of the
current state and performing state transitions as necessary.

When you need the update functionality of modules in state machines, you
should use the {\tt during()} functions of states. The during function of
the current state is called every time the state machine module itself is
updated.
\end{enumerate}

\section{Hardware Access}

\begin{enumerate}

\item{What are the resolutions of MMReadTime() and MMReadUTime()?}\par
The resolutions of both depend on the operating system as well as the
actual timer hardware on the CPU board. On QNX, {\tt MMReadTime()} has a
resolution equal to the system timer, which in RHexLib is set to 1
milliseconds. In contrast, {\tt MMReadUTime()} uses timers with the highest
available resolution. In QNX, on a 486, it has 0.2 microseconds or so
resolution whereas on a Pentium, it uses the CPU's instruction counter and
has 0.003 microseconds resolution.

\item{How do I decide when to add a new component to the Hardware class and
    what its interface should be?}\par
\end{enumerate}

\section{How do I do ...?}

\begin{enumerate}
\item
\end{enumerate}

\section{C++ Related}

The questions in this section relate to general C++ issues. A good C++ book
will always be a better reference than this section.

\begin{enumerate}
\item{Why is there no destructor for some classes?}\par
You need to define a destructor for a class, if there are explicit
deallocations or cleanup operations that you need to do. Usually, if you
have allocated any memory in the constructor, you need to deallocate them in
the destructor. For hardware related classes, the destructor usually
uninitializes the hardware and leaves it in a clean state. If you do not
need such operations, you do not need to define a destructor. Deletion of
the class object itself is done by code generated by the compiler.

\item{What is the {\tt virtual} keyword I see in some of the class
    definitions? What is the meaning of {\tt virtual mymethod( void ) = 0}?}
Virtual functions are a very useful feature of C++. They allow you to
override the method definitions of a class by its derived classes. This way,
if a derived class wants to provide a better, or more specific
implementation of a method which is already defined by its parent class,
this possible if the parent class had defined its method to be "virtual".

The {\tt = 0} at the end of a virtual method declaration defines the
function to be {\em pure virtual}. This is a special virtual function, in
that the parent class does not define what this function does {\em at
  all}. It only declares the arguments, the name, and the return value of
the method and leaves it up to its derived classes to define the
function. This is a very useful way for a class to define a {\em uniform 
  interface}, rather than an implementation. For example, the \Module\ class
has many pure virtual functions, which {\em must} be defined by all the
derived module classes, enforcing a uniform interface for all of them.

\item{What are abstract base classes?}
A class which has at least one pure virtual function as one of its methods
is an abstract base class. Instances of abstract base classes cannot be
created (i.e. the statement {\tt new Module} would fail to compile) because
some of their methods are left undefined. It is impossible to create an
object whose methods are still undefined. These classes only become useful
when you derive a class from them. That is why they are called (abstract)
base classes.
\end{enumerate}

\section{Common Pitfalls}

\begin{enumerate}
\item
\end{enumerate}

