%
% This file is part of RHexLib, 
%
% Copyright (c) 2001 The University of Michigan, its Regents,
% Fellows, Employees and Agents. All rights reserved, and distributed as
% free software under the following license.
% 
%  Redistribution and use in source and binary forms, with or without
% modification, are permitted provided that the following conditions are
% met:
% 
% 1) Redistributions of source code must retain the above copyright
% notice, this list of conditions, the following disclaimer and the
% file called "CREDITS" which accompanies this distribution.
% 
% 2) Redistributions in binary form must reproduce the above copyright
% notice, this list of conditions, the following disclaimer and the file
% called "CREDITS" which accompanies this distribution in the
% documentation and/or other materials provided with the distribution.
% 
% 3) Neither the name of the University of Michigan, Ann Arbor or the
% names of its contributors may be used to endorse or promote products
% derived from this software without specific prior written permission.
% 
% THIS SOFTWARE IS PROVIDED BY THE COPYRIGHT HOLDERS AND CONTRIBUTORS
% "AS IS" AND ANY EXPRESS OR IMPLIED WARRANTIES, INCLUDING, BUT NOT
% LIMITED TO, THE IMPLIED WARRANTIES OF MERCHANTABILITY AND FITNESS FOR
% A PARTICULAR PURPOSE ARE DISCLAIMED. IN NO EVENT SHALL THE REGENTS OR
% CONTRIBUTORS BE LIABLE FOR ANY DIRECT, INDIRECT, INCIDENTAL, SPECIAL,
% EXEMPLARY, OR CONSEQUENTIAL DAMAGES (INCLUDING, BUT NOT LIMITED TO,
% PROCUREMENT OF SUBSTITUTE GOODS OR SERVICES; LOSS OF USE, DATA, OR
% PROFITS; OR BUSINESS INTERRUPTION) HOWEVER CAUSED AND ON ANY THEORY OF
% LIABILITY, WHETHER IN CONTRACT, STRICT LIABILITY, OR TORT (INCLUDING
% NEGLIGENCE OR OTHERWISE) ARISING IN ANY WAY OUT OF THE USE OF THIS
% SOFTWARE, EVEN IF ADVISED OF THE POSSIBILITY OF SUCH DAMAGE.

%%%%%%%%%%%%%%%%%%%%%%%%%%%%%%%%%%%%%%%%%%%%%%%%%%%%%%%%%%%%%%%%%%%%%%
% $Id: hardware.tex,v 1.4 2001/07/18 20:04:50 ulucs Exp $
%
% Created       : Uluc Saranli, 01/06/2001
% Last Modified : Uluc Saranli, 06/27/2001
%
%%%%%%%%%%%%%%%%%%%%%%%%%%%%%%%%%%%%%%%%%%%%%%%%%%%%%%%%%%%%%%%%%%%%%%

\chapter{Low Level Hardware Interface}
\label{sec:hardware_interface}

To facilitate the use of the library on more than one hardware platforms,
\rhexlib\ enforces a uniform interface to possibly different low level
hardware interfacing functions. Different platforms which have similar
components such as encoders, analog inputs/outputs, digital inputs/outputs
and timers, possibly with different connection schemes and access
mechanisms, can be accessed through the virtual \Hardware\
class. Instantiations of this class can, hence, hide all the details of the
low level hardware by providing a standard set of member functions.

\section{The \Hardware\ Class}
\label{sec:hardware_class}

\begin{codesegment}
#include "Hardware.hh"
\end{codesegment}

\begin{classdef}
class Hardware {
public:

  virtual CLOCK readClock( void ) = 0;
  virtual CLOCK readUClock( void ) = 0;
  virtual void driveEnable( uint index, bool enable ) = 0;

  virtual void getParams( uint index, HWParam_t *param ) = 0;
  virtual void getCalibration( uint index, HWCalib_t *calib ) = 0;
  virtual void setCalibration( uint index, HWCalib_t *calib ) = 0;

  EncoderHW   * encoders;
  AnalogHW    * analogIO;
  DigitalHW   * digitalIO;
  TimerHW     * timers;
  AccelHW     * accels;
  GyroHW      * gyros;
  PowerHW     * power;
  SwitchHW    * switches;
  DialHW      * dials;
  DCMotorHW   * dcmotors;
};
\end{classdef}

\begin{prototype}
CLOCK Hardware::readClock( void );
\end{prototype}

The inherited class must implement this method to return the value of the
low resolution clock provided by the hardware. This function relies on the
operating system clock and and is computationally inexpensive. Consequently,
this function should be used for most clock reads, unless higher resolution
is absolutely necessary.  In that case, the {\tt Hardware::readUClock()}
function may be used.

\begin{prototype}
CLOCK Hardware::readUClock( void );
\end{prototype}

The inherited class must implement this method to return the value of the
high resolution clock provided by the hardware. This function is specified
to have a resolution higher than or equal to that of the {\tt readClock() }
method. It is usually implemented through hardware timers and takes
substantially more time than its low resolution counterpart. It is provided
mainly for tasks requiring very low timing granularity.

\begin{prototype}
void Hardware::driveEnable( uint index, \bool\ enable );
\end{prototype}

The inherited class must implement this method to enable or disable,
depending on the value of {\tt enable}, the motor drives associated with the
motor axis specified by {\tt index}.

{\bf Note:} Some hardware implementations may not support individual
commands for different axes. In this case, the effect of enabling any motor
drive will enable all the drives and they will remain so until all the
previously enabled drives are disabled.

\begin{prototype}
void Hardware::getParams( uint index, \hwparam\ *param );
\end{prototype}

The inherited class must implement this method to retrieve the constant
parameters of the hardware object associated with the indicated axis (see
the \hwparam\ data type in \S\ref{sec:special_types}). These parameters are
static properties of the underlying hardware and do not change through the
execution of the program (e.g. torque constant, speed constant, gear ratio
etc. ).

\begin{prototype}
void Hardware::getCalibration( uint index, \hwcalib *calib ); \\
void Hardware::setCalibration( uint index, \hwcalib *calib );
\end{prototype}

These methods are depracated and are kept for backwards compatibility. The
inherited class must forward these function calls to the corresponding
methods of the \DCMotorHW\ class. Modules should also use the \DCMotorHW\
interface instead of this.

\begin{prototype}
EncoderHW   * encoders; \\
AnalogHW    * analogIO; \\
DigitalHW   * digitalIO; \\
TimerHW     * timers; \\
AccelHW     * accels; \\
GyroHW      * gyros; \\
PowerHW     * power; \\
SwitchHW    * switches; \\
DialHW      * dials; \\
DCMotorHW   * dcmotors;
\end{prototype}

These fields point to various components provided by the particular
instantiation of the hardware. Each of these components provides its own
interface to access its functionality (see the following sections).

These pointers are initialized by the Hardware base class constructor to be
{\tt NULL} by default. When the instantiation of a particular Hardware
inherited class wants provide the functionality of any of these components,
it allocates an object of the corresponding type (which also need to be an
inherited class), and assigns its pointer to one of these fields. If, on the
other hand, the hardware instantiation does not support any of these
components, the corresponding pointer should be left as {\tt NULL}

\section{The \EncoderHW\ Class}
\label{sec:encoder_class}

The low level interface to incremental encoders in the system is implemented 
through the \EncoderHW\ class. 

\begin{classdef}
class EncoderHW {
public:
  virtual void   enable( uint axis ) = 0;
  virtual void   disable( uint axis ) = 0;

  virtual uint16 read( uint axis ) = 0;
  virtual void   reset( uint axis ) = 0;
};
\end{classdef}

\begin{prototype}
void EncoderHW::enable( uint axis ); \\
void EncoderHW::disable( uint axis );
\end{prototype}

The inherited class must implement these methods to enable and disable the
encoder hardware for the indicated axis.

\begin{prototype}
uint16 EncoderHW::read( uint axis ); \\
void   EncoderHW::reset( uint axis );
\end{prototype}

The inherited class must implement these methods to read and reset the
current 16 bit encoder count of the indicated axis.

\section{The \AnalogHW\ Class}
\label{sec:analog_class}

\begin{classdef}
class AnalogHW {
public:
  virtual float read( uint channel ) = 0;
  virtual void  write( uint channel, float value ) = 0;
};
\end{classdef}

\begin{prototype}
float AnalogHW::read( uint channel );
\end{prototype}

The inherited class must implement this method to read the analog input from
the indicated channel. The time resolution of the analog inputs depend on
the particular implementation of the AnalogHW class.

\begin{prototype}
float AnalogHW::write( uint channel );
\end{prototype}

The inherited class must implement this method to output the supplied
voltage on the indicated analog output channel.

\section{The \DigitalHW\ Class}
\label{sec:digital_class}

The \DigitalHW\ interface organizes the digital inputs and outputs into
groups of 8 bits. The interface class methods can be used to access a
particular byte in its entirety or an individual bit of a particular
byte. Note that the actual hardware digital lines that each of these bits
correspond, as well as the I/O directionality of the lines depends on the
specific implementation.

\begin{classdef}
class DigitalHW {
public:
  virtual uint   getByte( uint byte ) = 0;
  virtual void   setByte( uint byte, uint8 value ) = 0;
  virtual bool   getBit( uint byte, uint bit ) = 0;
  virtual void   setBit( uint byte, uint bit, bool value ) = 0;
};
\end{classdef}

\begin{prototype}
uint DigitalHW::getByte( uint byte );
\end{prototype}

The inherited class must implement this method to input a byte from the
digital input hardware. Note that if any of the lines are configured in
hardware to be outputs, then the returned value is unspecified.

\begin{prototype}
void DigitalHW::setByte( uint byte, uint8 value );
\end{prototype}

The inherited class must implement this method to output a byte to the
digital output hardware. Note that if any of the corresponding bits are
configured in hardware to be inputs, then this method does not effect the
bit.

\begin{prototype}
bool DigitalHW::getBit( uint byte, uint bit );
\end{prototype}

The inherited class must implement this method to input a bit from the
digital input hardware. Note that if the corresponding digital line is
configured in hardware to be an output, then the returned value is
unspecified.

\begin{prototype}
void DigitalHW::setBit( uint byte, uint bit, bool value );
\end{prototype}

The inherited class must implement this method to output a bit to the
digital output hardware. Note that if the corresponding bit is configured in
hardware to be an input, then this method has no effect.

\section{The \TimerHW\ Class}
\label{sec:timer_class}

The \TimerHW\ interface is used to communicate with the Intel 82C54 timers
in the system. Please consult the 82C54 datasheets for specifications of
different timer modes and a desription of their operation.

\begin{classdef}
class TimerHW {
public:
  virtual void   setup( uint timer, uint mode, uint32 hertz ) = 0;
  virtual uint16 read( uint timer ) = 0;
  virtual void   getInfo( uint timer, uint *initial, uint *clockFreq ) = 0;
};
\end{classdef}

\begin{prototype}
void setup( uint timer, uint mode, uint32 hertz );
\end{prototype}

The inherited class must implement this method to configure the indicated
timer with the given mode and frequency. The actual frequency of operation
depends on the hardware clock frequency of the timer chips.

\begin{prototype}
uint16 read( uint timer );
\end{prototype}

The inherited class must implement this method to read the current 16 bit
counter value of the indicated timer.

\begin{prototype}
void getInfo( uint timer, uint *initial, uint *clockFreq );
\end{prototype}

The inherited class must implement this method to return the initial count
that is configured through the {\tt setup()} method call as well as the
frequency of the hardware clock used by the timer chips.

\section{The \AccelHW\ Class}
\label{sec:accel_class}

\begin{classdef}
class AccelHW {
public:

  typedef enum { AXIS_X, AXIS_Y, AXIS_Z } Axis;

  virtual float read( Axis a ) = 0;
  virtual void  getInfo( Axis axis, float *resolution, float *limit ) = 0;
};
\end{classdef}

\section{The \GyroHW\ Class}
\label{sec:gyro_class}

\begin{classdef}
class GyroHW {
public:
    
  typedef enum { AXIS_X, AXIS_Y, AXIS_Z } Axis;
    
  virtual float read( Axis a ) = 0;
  virtual void  getInfo( Axis axis, float *resolution, float *limit ) = 0;
};
\end{classdef}

\section{The \PowerHW\ Class}
\label{sec:power_class}

\begin{classdef}
class PowerHW {
public:

  virtual float voltage( void ) = 0;
  virtual float current( void ) = 0;
};
\end{classdef}

\section{The \SwitchHW\ Class}
\label{sec:switch_class}

The \SwitchHW\ is defined to provide transparent access to switch components 
such as DIP switches or pushbuttons.

\begin{classdef}
class SwitchHW {
public:
  virtual bool read( uint index ) = 0;
};
\end{classdef}

\begin{prototype}
bool read( uint index );
\end{prototype}

The inherited class must implement this method to return a flag indicating
whether a switch is ON or OFF.

\section{The \DialHW\ Class}
\label{sec:dial_class}

The \DialHW\ interface is designed to provide uniform access to various
forms of continuous inputs, such as remote control sticks, potentiometer
dials etc.

\begin{classdef}
class DialHW {
public:
  virtual float read( uint index ) = 0;
};
\end{classdef}

\begin{prototype}
float read( uint index );
\end{prototype}

The inherited class must implement this method to return a number in the
range $[-1, 1]$ corresponding to the current value of the indicated
dial. Note that the hardware component corresponding to a particular index
is determined by the specific instantiation of the hardware object.

\section{The \DCMotorHW\ Class}
\label{sec:dcmotor_class}

The \DCMotorHW\ class is designed to interface all aspects of DC motors
in the system. These range from setting the terminal voltage of the motor to 
reading either measured or estimated motor current, voltage, temperature etc.

\begin{classdef}
class DCMotorHW {
public:
  typedef enum { VOLTAGE_MODE, CURRENT_MODE } ControlMode;
  
  virtual float getVoltage( uint index ) = 0;
  virtual float getCurrent( uint index ) = 0;
  virtual float getBackEMF( uint index ) = 0;
  virtual float getTemperature( uint index ) = 0;

  virtual ControlMode getControlMode( uint index ) = 0;
  virtual void setCommand( uint index, float cmd ) = 0;
  virtual float getCommand( uint index ) = 0;

  virtual void getCalibration( uint index, HWCalib_t *calib ) = 0;
  virtual void setCalibration( uint index, HWCalib_t *calib ) = 0;
  
};
\end{classdef}

\begin{prototype}
float getVoltage( uint index ); \\
float getCurrent( uint index ); \\
float getBackEMF( uint index ); \\
float getTemperature( uint index );
\end{prototype}

The inherited class must implement these methods to return several
quantities associated with a particular DC motor in the system. Note that
the hardware interface does not specify whether these values are actually
measured or estimated based on available data. Specific hardware
implementations will describe the details of how these values are computed.

\begin{prototype}
DCMotorHW::ControlMode getControlMode( uint index );
\end{prototype}

The inherited class must implement this method to return the control mode of
the underlying DC motor drive circuitry. Higher level motion controllers
should act according to whether the motors operate at {\tt
DCMotorHW::VOLTAGE\_MODE} or {\tt DCMotorHW::CURRENT\_MODE}.

\begin{prototype}
void setCommand( uint index, float cmd );
\end{prototype}

The inherited class must implement this method to set the command to the
indicated motor. The semantics of the command depend on which mode the motor
drives are operating at. In {\tt VOLTAGE\_MODE}, the supplied voltage value
is treated as the desired motor terminal voltage in Volts. In {\tt
CURRENT\_MODE}, the value is treated as the desired motor armature current
in Amperes.

\begin{prototype}
float getCommand( uint index );
\end{prototype}

The inherited class must implement this method to return the current motor
command for the indicated axis. The semantics of the returned value are as
described in the {\tt getCommand()} method.

\begin{prototype}
void getCalibration( uint index, HWCalib\_t *calib ); \\
void setCalibration( uint index, HWCalib\_t *calib );
\end{prototype}

The inherited class must implement these methods to return and set the
calibration information for the indicated motor axis (see the \hwcalib\ data
type in \S\ref{sec:special_types}). These functions are provided by the
\DCMotorHW\ class to facilitate access to calibration information by all the
components in the system. The low level hardware interface usually does not
make any use of this information.

Normally, when a \rhexlib\ program starts up, a calibration procedure should
be run (e.g. \CalibMachine\ module, see \S\ref{sec:calib_machine}) set the
calibration through the method {\tt setCalibration()}. Then, all the other
components in the system can query the low level hardware to access the
calibration information transparently.

