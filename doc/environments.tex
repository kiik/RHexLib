%
% This file is part of RHexLib, 
%
% Copyright (c) 2001 The University of Michigan, its Regents,
% Fellows, Employees and Agents. All rights reserved, and distributed as
% free software under the following license.
% 
%  Redistribution and use in source and binary forms, with or without
% modification, are permitted provided that the following conditions are
% met:
% 
% 1) Redistributions of source code must retain the above copyright
% notice, this list of conditions, the following disclaimer and the
% file called "CREDITS" which accompanies this distribution.
% 
% 2) Redistributions in binary form must reproduce the above copyright
% notice, this list of conditions, the following disclaimer and the file
% called "CREDITS" which accompanies this distribution in the
% documentation and/or other materials provided with the distribution.
% 
% 3) Neither the name of the University of Michigan, Ann Arbor or the
% names of its contributors may be used to endorse or promote products
% derived from this software without specific prior written permission.
% 
% THIS SOFTWARE IS PROVIDED BY THE COPYRIGHT HOLDERS AND CONTRIBUTORS
% "AS IS" AND ANY EXPRESS OR IMPLIED WARRANTIES, INCLUDING, BUT NOT
% LIMITED TO, THE IMPLIED WARRANTIES OF MERCHANTABILITY AND FITNESS FOR
% A PARTICULAR PURPOSE ARE DISCLAIMED. IN NO EVENT SHALL THE REGENTS OR
% CONTRIBUTORS BE LIABLE FOR ANY DIRECT, INDIRECT, INCIDENTAL, SPECIAL,
% EXEMPLARY, OR CONSEQUENTIAL DAMAGES (INCLUDING, BUT NOT LIMITED TO,
% PROCUREMENT OF SUBSTITUTE GOODS OR SERVICES; LOSS OF USE, DATA, OR
% PROFITS; OR BUSINESS INTERRUPTION) HOWEVER CAUSED AND ON ANY THEORY OF
% LIABILITY, WHETHER IN CONTRACT, STRICT LIABILITY, OR TORT (INCLUDING
% NEGLIGENCE OR OTHERWISE) ARISING IN ANY WAY OUT OF THE USE OF THIS
% SOFTWARE, EVEN IF ADVISED OF THE POSSIBILITY OF SUCH DAMAGE.

%%%%%%%%%%%%%%%%%%%%%%%%%%%%%%%%%%%%%%%%%%%%%%%%%%%%%%%%%%%%%%%%%%%%%%
% $Id: environments.tex,v 1.4 2001/07/19 16:35:56 ulucs Exp $
%
%  This file defines LaTeX environments for various styles in the documentation
%
% Created       : Uluc Saranli, 01/06/2001
% Last Modified : Uluc Saranli, 06/27/2001
%
%%%%%%%%%%%%%%%%%%%%%%%%%%%%%%%%%%%%%%%%%%%%%%%%%%%%%%%%%%%%%%%%%%%%%%

% ------------------------------------------------------- %
% Environments for data types -------------------------%

% Width of the boxes used around various environments below -------- %
\def\codeboxwidth{5.9in}

% Commands to encapsulate a part of text in a box ------------------ %
\newbox\TextBox
\def\starttextbox{\global\setbox\TextBox\hbox\bgroup\ignorespaces}
\def\stoptextbox{\egroup}

% Environment for outputting data type definitions ----------------- %
%\newenvironment{datatype}
%  {\begin{center}
%      \starttextbox
%        \begin{tabular}{l}} {
%        \end{tabular}
%      \stoptextbox
%      \framebox[5.9in][l]{\box\TextBox}
%    \end{center}
%}
\newenvironment{datatype}
  {\begin{center}
      \begin{tabular}{|p{6.0in}|}\hline} {
       \hline\end{tabular}
    \end{center}}

% New float type for Program Source code listings ------------------ %
\floatstyle{boxed}
\floatname{codefloat}{Listing}
\newfloat{codefloat}{thp}{code}

% Verbatim environment for code segments. Usually used inside codefloat ------ %
\newenvironment{codesegment}
  {\verbatim}
  {\endverbatim}

% Environment for outputting function prototypes ------------------- %
%\newenvironment{prototype}
%  {\vspace{0.15in}\begin{center}
%      \starttextbox\tt
%        \begin{tabular}{l}} {
%        \end{tabular}
%      \stoptextbox
%      \framebox[\codeboxwidth][l]{\box\TextBox}
%   \end{center}}
\newenvironment{prototype}
  {\begin{itemize}\item
      \tt} {\end{itemize}}

\newenvironment{classdef}{\small \verbatim} {\endverbatim}

% Environment for module description header ------------------------ %
\newenvironment{moduleheader}
  {\begin{center}
      \begin{tabular}{| >{\bf \hfill} p{1.2in}p{4.4in}|}
        \hline} {
        \hline
      \end{tabular}
    \end{center}}
\newcommand{\classname}[1]{Class Name: & #1 \\}
\newcommand{\modulebase}[1]{Base Classes: & #1 \\}
\newcommand{\modulename}[1]{Module Name: & "#1" \\}
\newcommand{\moduleflags}[1]{Module Flags: & #1 \\}
\newcommand{\usedmodules}[1]{Used Modules: & #1 \\}
\newcommand{\mline}{}
\newcommand{\constructors}{{\bf Class constructors:}}
\newcommand{\localinterface}{{\bf Local interface:}}
\newcommand{\datatypes}{{\bf Data types:}}
\newcommand{\configsymbols}{{\bf Configuration Symbols:}}
\newcommand{\symbolitem}[1]{\item{#1}\index{Configuration symbols!#1}}
\newcommand{\symbolindex}[1]{\index{Configuration symbols!#1}}
\newcommand{\examplecode}{{\bf Example code:}}

% Definitions of Synopsis and Details headings --------------------- %
\newcommand{\detailsheading}
 {\par\vspace{0.15in}\noindent\hspace{0.3in}
   {\bf \large Details}\par}

\newcommand{\synopsisheading}
 {\par\vspace{0.15in}\noindent\hspace{0.3in}
   {\bf \large Synopsis}\par}

% Some useful macros for include, define and typedef statements ---- %
\newcommand{\typedefcmd}[1]{{\tt typedef #1}}
\newcommand{\includecmd}[1] {{\tt \#include #1}}
\newcommand{\definecmd}[1] {{\tt \#define #1}}


