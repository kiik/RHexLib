%
% This file is part of RHexLib, 
%
% Copyright (c) 2001 The University of Michigan, its Regents,
% Fellows, Employees and Agents. All rights reserved, and distributed as
% free software under the following license.
% 
%  Redistribution and use in source and binary forms, with or without
% modification, are permitted provided that the following conditions are
% met:
% 
% 1) Redistributions of source code must retain the above copyright
% notice, this list of conditions, the following disclaimer and the
% file called "CREDITS" which accompanies this distribution.
% 
% 2) Redistributions in binary form must reproduce the above copyright
% notice, this list of conditions, the following disclaimer and the file
% called "CREDITS" which accompanies this distribution in the
% documentation and/or other materials provided with the distribution.
% 
% 3) Neither the name of the University of Michigan, Ann Arbor or the
% names of its contributors may be used to endorse or promote products
% derived from this software without specific prior written permission.
% 
% THIS SOFTWARE IS PROVIDED BY THE COPYRIGHT HOLDERS AND CONTRIBUTORS
% "AS IS" AND ANY EXPRESS OR IMPLIED WARRANTIES, INCLUDING, BUT NOT
% LIMITED TO, THE IMPLIED WARRANTIES OF MERCHANTABILITY AND FITNESS FOR
% A PARTICULAR PURPOSE ARE DISCLAIMED. IN NO EVENT SHALL THE REGENTS OR
% CONTRIBUTORS BE LIABLE FOR ANY DIRECT, INDIRECT, INCIDENTAL, SPECIAL,
% EXEMPLARY, OR CONSEQUENTIAL DAMAGES (INCLUDING, BUT NOT LIMITED TO,
% PROCUREMENT OF SUBSTITUTE GOODS OR SERVICES; LOSS OF USE, DATA, OR
% PROFITS; OR BUSINESS INTERRUPTION) HOWEVER CAUSED AND ON ANY THEORY OF
% LIABILITY, WHETHER IN CONTRACT, STRICT LIABILITY, OR TORT (INCLUDING
% NEGLIGENCE OR OTHERWISE) ARISING IN ANY WAY OUT OF THE USE OF THIS
% SOFTWARE, EVEN IF ADVISED OF THE POSSIBILITY OF SUCH DAMAGE.

%%%%%%%%%%%%%%%%%%%%%%%%%%%%%%%%%%%%%%%%%%%%%%%%%%%%%%%%%%%%%%%%%%%%%%
% $Id: fundamentals.tex,v 1.5 2001/07/24 02:06:29 ulucs Exp $
%
% Created       : Uluc Saranli, 01/06/2001
% Last Modified : Uluc Saranli, 06/27/2001
%
%%%%%%%%%%%%%%%%%%%%%%%%%%%%%%%%%%%%%%%%%%%%%%%%%%%%%%%%%%%%%%%%%%%%%%

\chapter{Fundamentals}
\label{sec:fundamentals}

\section{Basic Data Types}

This section describes the basic data types used in \rhexlib.

\begin{codesegment}
#include "types.hh"
\end{codesegment}

\begin{datatype}
\typedefcmd{\bool}; \\
\end{datatype}

The type \bool\ is the standard boolean type. Variables of this type can
only have the value {\tt true} or {\tt false}.

\begin{datatype}
\typedefcmd{\vast}; \\
\typedefcmd{\uvast}; \\
\end{datatype}

The types \vast\ and \uvast\ are 64 bit signed and unsigned integer types
respectively. In some platforms, these data types are implemented through
C++ classes and may not provide all the standard operators for integers.

\begin{datatype}
\typedefcmd{\intb}; \\
\typedefcmd{\uintb}; \\
\typedefcmd{\intw}; \\
\typedefcmd{\uintw}; \\
\typedefcmd{\intdw}; \\
\typedefcmd{\uintdw}; \\
\typedefcmd{\intqw}; \\
\typedefcmd{\uintqw}; \\
\typedefcmd{\uint}; \\
\end{datatype}

These are signed and unsigned integer types which are guaranteed to have the
indicated precision on all platforms.

\begin{datatype}
\typedefcmd{\CLOCK}; \\
\end{datatype}

The \CLOCK\ type is a a 64 bit signed integer datatype intended to represent
a time value in microseconds.

\begin{datatype}
\typedefcmd{\floats}; \\
\end{datatype}

The \floats\ type is a C++ {\tt struct} holding an array of a known number
of {\tt float}s. If {\tt x} is of type \floats\ then {\tt x.count} is the
number of elements in the array and {\tt x.f[0]} through {\tt x.f[count-1]}
are the elements of the array.

%laura: i added in this next one
\begin{datatype}
\typedefcmd{\strings}; \\
\end{datatype}

The \strings\ type is a C++ {\tt struct} holding an array of a known
number of {\tt char *}s. If {\tt y} is of type \strings\ then {\tt
y.count} is the number of elements in the array and {\tt y.s[0]}
through {\tt y.s[count-1]} are the elements of the array.

\section{Special Data Types} 
\label{sec:special_types}

This section describes data types and classes used in various components of
\rhexlib.

% eric: this is not very descriptive -e 
% uluc: well, I hadn't completed it yet...

\begin{codesegment}
#include "ModuleManager.hh"
\end{codesegment}

\begin{datatype}
\typedefcmd{\mmstep}; \\
\end{datatype}

The \mmstep\ is an integer type to represent the number of module manager
steps. The time period corresponding to a number of module manager steps
depends on the period of the module manager, which can be queried with the 
{\tt MMGetStepPeriod()} function (see \S\ref{sec:mm_info}).

\begin{codesegment}
#include "MotorControl.hh"
\end{codesegment}

\begin{datatype}
\typedefcmd{\motortarget}; \\
\end{datatype}

The \motortarget\ type is a C++ {\tt struct} holding the target
position, velocity and acceleration for a motor. The following fields for
\motortarget\ are defined in {\tt MotorControl.hh}:

\begin{itemize}
\item{{\tt double pos}: Target position ( rad )}
\item{{\tt double vel}: Target velocity ( rad/s )}
\item{{\tt double acc}: Target acceleration ( rad/s$^2$ )}
\end{itemize}

\begin{datatype}
\typedefcmd{\motorgains}; \\
\end{datatype}

The \motorgains\ type is a C++ {\tt struct} holding motor control
gains. The following fields for \motorgains\ are defined in {\tt
MotorControl.hh}:

\begin{itemize}
\item{{\tt double kp}: Proportional gain ( Nm/rad )}
\item{{\tt double kd}: Derivative gain ( Nm/(rad/s) )}
\item{{\tt double ka}: Acceleration gain ( Nm/(rad/s$^2$) )}
\end{itemize}

\section{Utility Macros} 
\label{sec:utility_macros}

This section describes several preprocessor macros that RHexLib defines for
common tasks.

\begin{codesegment}
#include "types.hh"
\end{codesegment}

\begin{datatype}
\VastToDouble( \CLOCK\ c ) \\
\VastToLong( \CLOCK\ c ) \\
\end{datatype}

These macros convert valus of type {\tt vast} to native C++ data types {\tt
  double} and {\tt long}, respectively. Note that conversion to {\tt long}
results in loss of information because {\tt vast} is a 64 bit type.

\begin{datatype}
\ClocksPerSec \\
\end{datatype}

This constant holds the number of clock ticks per second.

\begin{datatype}
\ClockToDouble( \CLOCK\ c ) \\
\ClockToLong( \CLOCK\ c ) \\
\end{datatype}

These macros convert valus of type \CLOCK\ to native C++ data types {\tt
  double} and {\tt long}, respectively. Note that conversion to {\tt long}
results in loss of information because \CLOCK\ is a 64 bit type.

\begin{datatype}
\ClockToSec( \CLOCK\ c ) \\
\ClockToMin( \CLOCK\ c ) \\
\ClockToHour( \CLOCK\ c ) \\
\SecToClock( double sec ) \\
\MinToClock( double min ) \\
\HourToClock( double hr ) \\
\end{datatype}

These macros convert \CLOCK\ values to seconds, minutes and hours and
vice-versa.

\section{Code Conventions}

